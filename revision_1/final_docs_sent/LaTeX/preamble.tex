%IDIOMAS:
\usepackage[spanish,english]{babel} %Poner es-tabla para que ponga los caption en español. El último idioma que se cargue es el que se queda por defecto. Luego en el texto se puede cambiar con \selectlanguage{idioma}.
\usepackage[utf8]{inputenc}	%Para los acentos

%PARA ESCRIBIR EN COLOR:
\usepackage{color}	%Para escribir en color con el comando \textcolor{color}{texto}
\usepackage[dvipsnames]{xcolor} %Paquete con colores predefinidos extra

%PARA USAR TIPOGRAFÍA DE CONJUNTOS
\usepackage{amssymb}

%PARA CAMBIAR EL TIPO DE FUENTE (LETRA):
%Por defecto tenemos Computer Modern (no es necesario cargar ningun paquete), pero podemos seleccionar otro tipo de letra.
%Cargamos un paquete para cada tipo de letra:
%\usepackage{times} %fontcode ptm
%\usepackage{palatino} %fontcode ppl
%\usepackage{bookman} %fontcode pbk
%\usepackage{helvet} %fontcode phv
\usepackage{courier} %fontcode pcr
%Tras cargar el paquete, podemos cambiar la letra haciendo: {\fontfamily{fontcode}\selectfont  texto}

%PARA DIVIDIR EL TEXTO EN COLUMNAS:
\usepackage{multicol}		%Para dividir el texto en columnas con \begin{multicols}{nº de columnas\end{multicols}
\setlength{\columnsep}{0.5cm} %Para elegir la separación entre las columnas.

%INDENTADO Y MÁRGENES:
%Para el indentado (sangría de los párrafos).
\usepackage{indentfirst}	%Para que indente el primer párrafo tras un título de sección.
\setlength{\parindent}{0.5cm}	%Tamaño de la sangría.
%Para eliminar la sangría de un párrafo hacemos: {\setlength{\parindent}{0pt} texto del párrafo}

%Para los márgenes:
\usepackage{anysize}	
%\marginsize{3cm}{3cm}{1.5cm}{1.5cm}	%Indicamos los márgenes {izquierda}{derecha}{arriba}{abajo}
%\usepackage[top=0cm, bottom=1.5cm, inner=2cm, outer=2cm, headheight=3cm, includeheadfoot]{geometry} %Indicamos los márgenes para encuadernación.
\usepackage[top=1.5cm, bottom=3cm, inner=2cm, outer=2cm]{geometry} %Indicamos los márgenes para encuadernación.


%GRÁFICOS (PAQUETES PARA INTRODUCIR IMÁGENES):
%\usepackage[pdftex]{graphicx}
\usepackage{graphicx}
\DeclareGraphicsExtensions{.png,.jpg,.pdf,.mps,.gif,.bmp,.eps}
\usepackage{glossaries}
\usepackage{tikz}
\let\ig\includegraphics 

%Uso de flotantes en imágenes y tablas:
\usepackage{float}
\usepackage{subfig}

%Para modificar los captions de imágenes y tablas:
\usepackage[margin=0cm,font=footnotesize]{caption} 

%Para el uso de psfrag:
\usepackage{psfrag}
\usepackage{pstool}
\usepackage{epsfig}
\usepackage{graphics}

%Para incluir figuras mezcladas (en paralelo) con el texto:
\usepackage{wrapfig}

%Paquetes para tikz
\usepackage{pgf,tikz}
\usepackage{mathrsfs}
% \usetikzlibrary{arrows}
% \usetikzlibrary{angles,calc,intersections,quotes,arrows.meta}
% \usetikzlibrary{arrows.new}
\newcommand{\degre}{\ensuremath{^\circ}}

%ENCABEZADOS Y PIES DE PÁGINA:
\usepackage{fancyhdr}
%\pagestyle{fancyplain}
%\headheight=60pt %para cambiar el tamaño del encabezado
%\lhead{\includegraphics[width=0.08\textwidth]{MARCA_UMA.png}}
%\chead{ANTEPROYECTO \\}
%\rhead{\includegraphics[width=0.2\textwidth]{logo_etsii.jpg}}
%\renewcommand{\headrulewidth}{1.5pt} % grosor de la línea de la cabecera
%\renewcommand{\footrulewidth}{1.5pt} % grosor de la línea del pie de página

%Explicación:
%\comando[pagina par]{pagina impar}
%Nota: para incluir una imagen en el encabezado o pie de página, en caso de incluirla dentro de los corchetes [], es necesario meter el comando \includegraphics[opciones]{figura} entre corchetes {}.

%Encabezados:
\pagestyle{fancyplain}
%\lhead[\thepage]{Documento de tipo {\em article}}
%\chead[]{}
%\rhead[Manual básico de \LaTeX]{\thepage}
%\renewcommand{\headrulewidth}{0pt}

%Ejemplo encabezado con figuras:
\pagestyle{fancyplain}
\lhead[]{}
\chead[]{}
\rhead[]{}
\renewcommand{\headrulewidth}{0pt}
%\lhead[{\includegraphics[width=0.1\textwidth]{logo_uc3m.eps}}]{\includegraphics[width=0.1\textwidth]{logo_uc3m.eps}}
%\chead[]{}
%\rhead[{\includegraphics[width=0.3\textwidth]{logo_nanotec.eps}}]{\includegraphics[width=0.3\textwidth]{logo_nanotec.eps}}
%\renewcommand{\headrulewidth}{1pt}

%Pie de pagina:
\lfoot[\thepage]{}
\cfoot[]{}
\rfoot[]{\thepage}
\renewcommand{\footrulewidth}{0pt}

%Primera pagina de un capitulo (solo book):
\fancypagestyle{plain}{
\fancyhead[L]{}
\fancyhead[C]{}
\fancyhead[R]{}
\fancyfoot[L]{}
\fancyfoot[C]{}
\fancyfoot[R]{}
\renewcommand{\headrulewidth}{0pt}
\renewcommand{\footrulewidth}{0pt}
}
\pagestyle{fancy}

%NUMERACIÓN DE LAS SECCIONES Y SU INCLUSIÓN EN EL ÍNDICE:
\setcounter{secnumdepth}{6} %Para que numere las subsubsecciones y los paragraph
\setcounter{tocdepth}{6} %Para que introduzca las subsubsecciones y los paragraph en el índice
% Do not indent paragraphs nor subparagraphs
\makeatletter
\renewcommand\paragraph{\@startsection{paragraph}{4}{\z@}%
                                      {-3.25ex\@plus -1ex \@minus -.2ex}%
                                      {0.0001pt \@plus .2ex}%
                                      {\normalfont\normalsize\bfseries}}
\renewcommand\subparagraph{\@startsection{subparagraph}{5}{\z@}%
                                      {-3.25ex\@plus -1ex \@minus -.2ex}%
                                      {0.0001pt \@plus .2ex}%
                                      {\normalfont\normalsize\bfseries}}
\makeatother

%Esto es para que numere las tablas, ecuaciones y figuras según el capítulo en el que estén (solo para tipo book).
%\numberwithin{table}{chapter}
%\numberwithin{equation}{chapter}
%\numberwithin{figure}{chapter}

%ENUMERACIONES:
\usepackage{enumerate}

%TABLAS:
\usepackage{booktabs}	% Para aumentar grosor de tablas.
\usepackage{multirow}	% Para poder unir filas y columnas en las tablas.
\usepackage{colortbl}	% Para colorear tablas.
%\usepackage{slashbox}	% Para colocar una línea diagonal dividiendo las celdas.
\usepackage{makecell}
\usepackage{array}
\usepackage{pdflscape}  % Para incluir páginas apaisadas compilando con pdflatex
\usepackage{lscape}     % Para incluir páginas apaisadas compilando con latex

%Para lineas más gruesas en tablas:
%\thickhline >> línea horizontal de espesor definido en height
%' >> para columna interior, línea vertical de espesor definido en width (en lugar de |c|, ponemos 'c') 
%! >> para primera columna a la izquierda, línea vertical de espesor definido en width (en lugar de |c|c|..., ponemos !c|c|...)
%? >> para primera columna a la izquierda, línea vertical de espesor definido en width (en lugar de ...|c|c|, ponemos ...|c|c?)
\makeatletter
\newcommand{\thickhline}{%
    \noalign {\ifnum 0=`}\fi \hrule height 2pt
    \futurelet \reserved@a \@xhline
}
\newcolumntype{'}{@{\hskip\tabcolsep\vrule width 2pt\hskip\tabcolsep}}
\newcolumntype{!}{@{\vrule width 2pt\hskip\tabcolsep}}
\newcolumntype{?}{@{\hskip\tabcolsep\vrule width 2pt}}
\makeatother

%Para columnas de ancho fija:
\newcolumntype{L}[1]{>{\raggedright\let\newline\\\arraybackslash\hspace{0pt}}m{#1}}
\newcolumntype{C}[1]{>{\centering\let\newline\\\arraybackslash\hspace{0pt}}m{#1}}
\newcolumntype{R}[1]{>{\raggedleft\let\newline\\\arraybackslash\hspace{0pt}}m{#1}}

%Para unir columnas:
%\multicolumn{nº de columnas}{alineación}{texto}
%Para unir filas:
%\multirow{nº de filas}{ancho}[desplazamiento vertical]{texto}
%Para colorear filas:
%\rowcolor[rgb]{0,1,1}Roque & Fort \\
%\rowcolor[cmyk]{0.5,0,0,0.1}Abraham & Lapuerta \\
%Para colorear columnas (se emplea al definir las columnas o con el multicolumn):
%\begin{tabular}{|>{\columncolor[rgb]{0.7,0,0.7}} c |
%Para definir un color:
%\definecolor{micolor}{rgb}{0,1,0.5}
%Para escalar la tabla al ancho de página:
%\resizebox{<ancho_deseado>}{!} {<entorno tabular que se desea escalar>}

%MATEMÁTICAS Y SÍMBOLOS:
%Para introducir formulas matemáticas alinedas (align y align*):
\usepackage{amsmath,bm}
%\usepackage{amsmath,amssymb,amsfonts,latexsym}

%Símbolo del euro:
\usepackage{eurosym} %Para escribir el símbolo del euro se emplea \euro.

%HIPERVÍNCULOS:
% \usepackage[linktocpage]{hyperref}
\usepackage[linktocpage,breaklinks=true]{hyperref}
\hypersetup{
colorlinks = true,
%linkcolor  = Red,
%citecolor  = Red,
linkcolor  = MidnightBlue,
citecolor  = OliveGreen,
urlcolor   = blue
}

%CREAR COMANDOS PROPIOS
%Creamos un comando (\courier) para cambiar la letra:
\newcommand*{\courier}{\fontencoding{\encodingdefault}\fontfamily{pcr}\fontseries{m}\fontshape{n}\selectfont}
\newcommand*{\couriernegrita}{\fontencoding{\encodingdefault}\fontfamily{pcr}\fontseries{b}\fontshape{n}\selectfont}
%Creamos un comando para poner el símbolo de marca registrada:
\newcommand*{\R}{$^{\textregistered}$}
%Creamos el comando parcial de primer orden:
\providecommand{\parcial}[2]{\frac{\partial #1}{ \partial #2}}
%Creamos el comando parcial de segundo orden:
\providecommand{\pparcial}[2]{\frac{\partial^2 #1}{ \partial #2 ^2}}

%PARA INTRODUCIR CÓDIGO DE MATLAB, FORTRAN:
\usepackage{listings}
%\usepackage[]{mcode}	% Paquete de formato especial para MATLAB.
%\lstinputlisting{nombre_del_documento.m}
\definecolor{mygreen}{RGB}{28,172,0} % color values Red, Green, Blue
\definecolor{mylilas}{RGB}{170,55,241}
\definecolor{mybrown}{RGB}{165,42,42}

% Para poner los captions de los códigos en español:
\renewcommand{\lstlistingname}{Código}
% Para poner los captions de los códigos en inglés:
%\renewcommand{\lstlistingname}{Code}
% Para poner el índice de códigos en español:
% Documento en español:
\renewcommand\lstlistlistingname{Índice de códigos}
% Documento en inglés:
%\renewcommand\lstlistlistingname{List of Codes}

%Para introducir código Matlab:
\lstdefinestyle{MATLAB}{
	language=Matlab,%
    %basicstyle=\color{red},
    breaklines=true,%
    morekeywords={matlab2tikz},
    keywordstyle=\color{black},%
    morekeywords=[2]{1},
    keywordstyle=[2]{\color{mylilas}},
    identifierstyle=\color{black},%
    stringstyle=\color{mylilas},
    commentstyle=\color{mygreen},%
    showstringspaces=false,%without this there will be a symbol in the places where there is a space
    numbers=left,%
    numberstyle={\tiny \color{black}},% size of the numbers
    numbersep=9pt, % this defines how far the numbers are from the text
    emph=[1]{for,end,break,if,elseif,else,switch,case,otherwise,while,function,global},emphstyle=[1]\color{blue}, %some words to emphasise  
    emph=[2]{all,on}, emphstyle=[2]\color{mylilas},
}

% Para introducir código Fortran:
\lstdefinestyle{FORTRAN}{
	language=[90]Fortran,
    %basicstyle=\color{red},
    breaklines=true,%
    morekeywords={matlab2tikz},
    keywordstyle=\color{mybrown},%
    morekeywords=[2]{1}, 
    keywordstyle=[2]{\color{mybrown}},
    identifierstyle=\color{black},%
    stringstyle=\color{mylilas},
    commentstyle=\color{blue},%
    showstringspaces=false,%without this there will be a symbol in the places where there is a space
    numbers=left,%
    numberstyle={\tiny \color{black}},% size of the numbers
    numbersep=9pt, % this defines how far the numbers are from the text
    emph=[1]{real,pointer},emphstyle=[1]\color{mygreen}, %some words to emphasise  
    emph=[2]{type}, emphstyle=[2]\color{mybrown},
}

%PARA INTRODUCIR ARCHIVOS PDF EN EL DOCUMENTO:
\usepackage[final]{pdfpages}
%\includepdf[pages=inicial-final]{nombre_del_documento}\usepackage[pdftex]{graphicx}
%NOTA: Es necesario compilar con pdfLaTeX, por lo que hay que cargar este paquete para que las imágenes en .eps no den problemas.
\usepackage{epstopdf}

%PARA INTRODUCIR TEXTO DE RELLENO:
\usepackage{lipsum}
%\lipsup[numero de parrafo inicial - numero de parrafo final]

%PARA VARIAS CITAS BIBLIOGRAFICAS SEPARADAS POR GUION
\usepackage{cite}
%HASTA AQUÍ EL PREÁMBULO DEL DOCUMENTO. AHORA COMIENZA EL DOCUMENTO EN SÍ.
%%%%%%%%%%%%%%%%%%%%%%%%%%%%%%%%%%%%%%%%%%%%%%%%%%%%%%%%%%%%%%%%%%%%%%%%%%%%%%%%%%%%%%%%%%%%%%%%%%%%%%%%%%
%%%%%%%%%%%%%%%%%%%%%%%%%%%%%%%%%%%%%%%%%%%%%%%%%%%%%%%%%%%%%%%%%%%%%%%%%%%%%%%%%%%%%%%%%%%%%%%%%%%%%%%%%%
